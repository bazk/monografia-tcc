\chapter{Introdução}
\label{cha:introducao}

A natureza apresenta uma infinidade de seres vivos das mais diferentes espécies e diversas características. Dentre eles destacam-se aqueles que tem capacidade de auto-organização em colônias a fim de superar os limites físicos e cognitivos individuais e então vencer obstáculos e dificuldades impostas no ambiente que habitam \cite{navarro2012introduction}. Podemos facilmente extrair algumas qualidades que tais colônias apresentam: flexibilidade, robustez, descentralização etc. Por exemplo, o comportamento de formação de caminho explora a comunicação entre os indivíduos e pode ser observado em diversas espécies de formigas e consiste na busca por alimento, recrutamento em massa e finalmente na formação de um caminho eficiente entre o alimento e a colônia.

A área denominada robótica de enxame tem como objetivo principal o projeto de agentes autônomos simples de maneira que um grupo formado por tais demonstre comportamentos coletivos a partir de interações entre um e o grupo e entre um e o ambiente \cite{sahin2005swarm}. O projeto do controlador de um agente autônomo apresenta grandes dificuldades \cite{nolfi1994howtoevolve}, no entanto outra área da robótica se propõe a resolver esse problema: robótica evolutiva. Na maior parte da literatura dessa área \cite{nelson2009fitness}, o controlador é representado por uma rede neural artificial e seu treinamento é tratado como um problema de otimização, daí o uso da computação evolutiva.

Esse trabalho tem como objetivo avaliar diferentes estratégias e algoritmos para síntese de um controlador para robôs autônomos em um enxame cuja finalidade é reproduzir o comportamento de formação de caminho.

Esta monografia está organizada em 7 capítulos. O Capítulo 2 define os conceitos básicos da robótica evolutiva e as áreas relacionadas: redes neurais artificiais e computação evolutiva. O Capítulo 3 apresenta uma visão geral da robótica de enxame e define o problema de formação de caminho. O Capítulo 4 define a instância do problema avaliado nesse trabalho e a abordagem utilizada. O Capítulo 5 apresenta os detalhes de implementação e parâmetros utilizados e os experimentos realizados. O Capítulo 6 apresenta os resultados e análise dos experimentos. O Capítulo 7 apresenta considerações finais e trabalhos futuros.