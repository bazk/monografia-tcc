\chapter{Computação Bioinspirada}
\label{bioinspirada}

\section{Introdução}

introdução

\section{Redes Neurais Artificiais}

Redes Neurais Artificias (RNA) são sistemas computacionais inspirados em sistemas
nervosos animais capazes de aprendizado. Matematicamente, RNAs são aproximadores
universais \cite{hornik89universal}. Um RNA é composto por uma rede de unidades
de processamento simples (neurônios). Cada neurônio possui um sinal de saída e
pode receber um ou mais sinais de entrada.

\subsection{Multi-Layer Perceptron (MLP)}

Uma rede neural MLP é composta por uma série de neurônios organizados em camadas, onde cada neurônio recebe, como entrada, a saída dos neurônios da camada anterior. A saída (y) de um neurônio é determinada pela aplicação de uma função de ativação \(\theta\) sobre a combinação linear de todas as suas \(n\) entradas \((x_1, x_2, ..., x_n)\). Matematicamente,

\[ y = \theta ( \sum_{i=1}^{n} w_i x_i ) \]

onde \(w_i\) é um peso associado a entrada \(i\).

Diversas funções pode assumir o papel de função de ativação, sendo as mais comuns:
a função degrau [formula] e a função sigmóide [formula].

formulas

figure

Uma rede com uma camada intermediária é capaz de aproximar qualquer função
contínua. Com duas camadas intermediárias, qualquer função pode ser aproximada
\cite{cybenko89mlp}.

O ajuste dos pesos w1..wn é chamado treinamento e existe em duas formas:

\begin{description}
    \item[Supervisionado]: É fornecido ao algoritmo de treinamento um conjunto de entradas/saídas. O algoritmo iterativamente ajusta os pesos da rede a fim de que, dadas entradas do conjunto de treinamento, as saídas da rede aproximem-se das saídas.
    \item[Não supervisionado]: Não demanda conjunto de treinamento, os pesos
são ajustados considerando a aptidão da rede à solução do problema.
\end{description}

\subsection{Time-Delay Neural Network (TDNN)}

Extendendo os conceitos das redes MLP, uma TDNN permite à cada neurônio armazenar um
histórico dos sinais de entrada. Isto permite que a rede ganhe sensibilidade à padrões temporais, ou seja, a rede pode adaptar-se não só à padrões como também à sequência de padrões \cite{kaiser94tdnn}.

Este modelo de rede neural é de especial importância na área de robótica. As ações determinadas pelo controlador de um robô devem levar em conta o histórico sensorial e não apenas o estado sensorial atual.

\section{Controle de Robôs Móveis por RNA}

% ann as robot controller

% [zhang 2000]
% Although many types of neural networks can be used
% for classification purposes [105], our focus nonetheless is on
% the feedforward multilayer networks or multilayer perceptrons
% (MLPs) which are the most widely studied and used neural net-
% work classifiers.

\section{Algoritmos de Otimização para Treinamento de RNA}
\label{optimization-algorithms}

ga, pso, dpso, bpso, pga