\chapter{Rob{\'o}tica de Enxame}
\label{swarm}

\section{Introdução}

Observando os seres vivos presentes na natureza, podemos facilmente extrair algumas qualidades que estes apresentam: flexibilidade, robustez, descentralização, auto-organização. Em grande parte, o motivo por trás dessas qualidades é a coletividade. Por exemplo, em colônias de insetos sociais, diversos indivíduos se auto-organizam para realizar tarefas cuja  escala e complexidade superam, e muito, os limites físicos e cognitivos de cada indivíduo. Podemos visualizar uma colônia como um único super-organismo [Trianni 2011] e podemos interpretar inteligência como uma característica que emerge das interações entre componentes simples e interdependentes de um sistema.

Sob esse ponto de vista, uma série de algoritmos computacionais têm sido propostos em
uma área de pesquisa denominada inteligência coletiva (\textit{swarm intelligence}) [Bonabeau 1999]
[Kennedy 2001]. De forma análoga, robótica de enxame (\textit{swarm robotics}) também utiliza dos mesmos conceitos para resolução de diversos problemas da área de robótica.

Diversos algoritmos bioinspirados de inteligência coletiva vem sendo propostos [Bonabeau 1999] [Kennedy 2001] e avaliados no contexto da robótica de enxame [Navarro 2013]. Com eles é possível produzir vários comportamentos coletivos. Alguns relativamente simples e outros bastante mais complexos, que em geral dependem da execução coordenada de alguns dos comportamentos coletivos simples.

O comportamento de agregação, por exemplo, consiste simplesmente na reunião dos
robôs do enxame e é utilizado por outros comportamentos complexos, como o movimento
coletivo, a auto-montagem e a formação de padrões, nos quais em determinados momentos o
enxame se reúne.

No comportamento de dispersão, o enxame se distribui de modo a ocupar a maior área
possível do espaço sem perder comunicação. É útil em tarefas que envolvam a exploração
coletiva de um território desconhecido.

A formação de padrões é um comportamento em que os robôs devem movimentar-se de
maneira coordenada para distribuírem-se no espaço segundo um determinado modelo de
posicionamento.

Já no comportamento de movimento coletivo o enxame deve mover-se em conjunto e de
maneira coesa [Sperati 2008]. Há, nesse caso, uma clara inspiração na natureza, por exemplo na
maneira como se movimentam pássaros em um bando ou peixes em um cardume. Pelo menos
dois tipos de movimento coletivo podem ser identificados: formation, em que o posicionamento e a
orientação dos robôs mantêm-se fixa, e flocking, em que isso não acontece.

O comportamento de busca por alimento (foraging) observado na natureza pode inspirar
uma gama mais ampla de problemas em que os robôs devem encontrar as coordenadas de um
ponto do ambiente com características definidas.

A formação de caminho é uma variação de movimento coletivo que também se enquadra em busca por alimento. Visa o trajeto do enxame pelo menor caminho entre dois pontos, assim como, na natureza, as formigas utlizam de feromônios para encontrar e navegar pelo menor caminho entre o formigueiro e alguma fonte de alimento.

Alocação de tarefas é um comportamento genérico que se aplica a diversos problemas em
que o enxame precisa, de maneira coletiva e descentralizada, definir a tarefa que cabe a cada
robô. Trata-se de um problema relativamente complexo em que diversos estudos vêm sendo
realizados.

O transporte coletivo de objetos é outro comportamento facilmente observável na natureza,
por exemplo em colônias de formigas. Envolve alto grau de coordenação e depende de vários
dos comportamentos básicos mais simples [Gross 2006].

Um comportamento com importante aplicação na exploração de ambientes desconhecidos
é o do mapeamento coletivo, em que o comportamento de dispersão é associado à troca de
informações entre robôs com o objetivo de produzir uma representação mais ampla do ambiente.

Outro comportamento relativamente complexo é a auto-montagem [Christensen 2007], que
consiste na agregação e interconexão de robôs formando padrões que podem ser posteriormente
utilizados para a solução coletiva de problemas.

\section{Formação de caminho}



% - Deslocamento coordenado (flocking) (Sperati 08 e trabalhos relacionados)
% - Deslocamento coordenado (flocking) para solução coletiva de problemas
%   (abordagem proposta pelo Sperati 11, foraging, transporte coletivo e
%   trabalhos relacionados)