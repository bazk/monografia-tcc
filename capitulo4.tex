\chapter{Implementação}
\label{implementacao}

\section{Introdução}

\section{Ambiente de simulação}



\cite{miglino96evolving}
There are several reasons why those who want to use computer models to develop control
systems for real robots may encounter problems:
(a) Numerical simulations do not usually consider all the physical laws of the interaction of
a real agent with its own environment, such as mass, weight, friction, inertia, etc.
(b) Physical sensors deliver uncertain values, and commands to actuators have very
uncertain effects, whereas simulative models often use grid-worlds and sensors which return
perfect information.
(c) Different physical sensors and actuators, even if apparently identical, may perform
differently because of slight differences in the electronics and mechanics or because of their
different positions on the robot. This fact is usually ignored in computer models.



% Implementação
% - Introdução
% - Simulação (box2d, atual)
% - Execução em paralelo (OpenCL)
% - Ferramenta para acompanhamento de experimentos