\chapter{Implementação}
\label{implementacao}

\section{Introdução}

\section{Os robôs}

O robô utilizado como referência é o \textit{e-puck} \cite{mondada2009epuck}, um pequeno robô cilíndrico com 7 centímetros de diâmetro. Dois motores independentes controlam duas rodas provendo tração diferencial e velocidade de até 8.2 cm/s. Oito sensores de luz infravermelha ao redor do robô permitem a detecção de obstáculos próximos (até aproximadamente 2.5 cm). Mais um sensor de luz infravermelha é colocado em baixo do robô para detecção da cor do chão (usado para diferenciar as áreas alvo do resto do ambiente). Além disso, o robô possuí uma câmera com alcance de 35 cm e campo de visão de $144\,^{\circ}$. Dois LEDs de cores diferentes na dianteira e traseira do robô, podendo ser acessos ou apagados, são detectáveis pela câmera de outros robôs permitindo uma forma de comunicação.

Para representar a informação captada pela câmera, codificamos, em quatro valores binários, a presença de pontos azuis ou vermelhos (correspondendo aos LEDs dianteiros e traseiros, respectivamente) na região direita ou esquerda do campo de visão.

Como Sperati et al. [\cite{sperati2011path}], abordamos o problema de formação de caminho utilizando robótica evolutiva. Cada robô é controlado por uma rede neural artificial (\ref{fig-}) cuja entrada é composta por todos os sensores normalizados e a saída controla os dois motores e os dois LEDs. Vale observar que a configuração da rede (estrutura e pesos) é homogênea à todos os robôs do enxame.

% figure here

Matematicamente, a ativação $O_{j}$ do neurônio de saída $j$ no instante $t$ é computado pelas seguintes equações:

$$
O_{j}(t) = \sigma (\sum_{i} w_{ij}^{OI} I_{i}(t) + \sum_{i} w_{ij}^{OH} H_{i}(t) + \beta_{j}^{O})
$$
$$
H_{j}(t) = \tau_{j} \sigma (\sum_{i} w_{ij}^{HI} I_{i}(t) + \beta_{j}^{H}) + (1 - \tau_{j}) H_{j} (t - 1)
$$

onde $I_{i}$ é o valor da entrada $i$ (sensor normalizado). $\beta_{j}^{H}$ e $\beta_{j}^{O}$  representam o \textit{bias} relativo ao neurônio $j$ da camada intermediária e de saída, respectivamente. Finalmente, $w_{ij}^{OI}$, $w_{ij}^{OH}$ e $w_{ij}^{HI}$ são os pesos relativos a entrada $i$ e saída $j$ das sinapses que ligam, respectivamente, os neurônios: de entrada para saída, de entrada para intermediários e intermediários para saída.

% >>>> \tau_{j}

\section{O ambiente físico}

Os robôs são colocados em um espaço quadrado (250 cm de lado) delimitado por paredes. Dentro desse espaço, duas circunferências (16 cm de raio), denotadas por uma coloração escura (em contraste ao restante do espaço que apresenta coloração mais clara), representam as áreas alvo. O sensor de luz infravermelha localizado em baixo do robô é sensível à diferença de coloração, podendo assim perceber se está sobre uma área alvo ou não. Adicionalmente, no centro das áreas alvo, dois LEDs vermelhos sempre acessos são visíveis, indistinguivelmente, como os LEDs na traseira dos robôs.

\section{Treinamento}
\label{sec:training}

O treinamento -- não supervisionado -- da rede pode ser visto da seguinte forma: queremos sintetizar uma solução para o problema de formação de caminho, ou seja, queremos encontrar parâmetros (conjunto de pesos, \textit{bias} e \textit{time constraints}) tal que um enxame de robôs, ao aplicar tais parâmetros homogeneamente a rede neural de cada um dos robôs, apresente comportamento esperado. Assim, trataremos esse como um problema de otimização e utilizaremos os algoritmos apresentados na seção \ref{}.

Para isso, precisamos definir uma função de \textit{fitness}. Conforme \cite{sperati2011path}, a \textit{fitness} $F$ de uma solução $S$ é computada a partir de um teste com os robôs no ambiente físico por um determinado intervalo de tempo $T$. Dividimos esse intervalo em duas partes $T_{a}$ e $T_{b}$. Durante a primeira parte, a \textit{fitness} não é avaliada. Essa avaliação só começa a acontecer durante a segunda parte do teste.

No início do teste, cada robô $i$ ganha um valor virtual de energia $e_{i} = 2$. A função $\delta$ aproxima o consumo de energia a fim de que um robô, a máxima velocidade, consuma exatamente uma unidade de energia para mover-se de uma área alvo a outra.

$$
\delta_{i} (t) = \frac{( | \omega_{ir} (t) | + | \omega_{il} (t) |) r}{d}
$$

onde $r$ é o raio das rodas do robô, $d$ é a distância entre as áreas alvo e $\omega_{ir} (t)$ e $\omega_{il} (t)$ equivalem a velocidade angular das rodas direita e esquerda, respectivamente, do robô $i$ no momento $t$.

Então, os seguintes passos são executados:

\begin{enumerate}
    \item A solução $S$ é aplicada ao enxame de robôs.
    \item Prepara-se o ambiente de testes, ou seja, as áreas alvo e os robôs são colocados em posições aleatórias dentro do ambiente.
    \item Os robôs são ligados e estão livres para mover-se dentro do ambiente sem afetar a \textit{fitness} durante a primeira parte do teste ($T_{a}$).
    \item Em seguida, inicia-se a segunda parte do teste ($T_{b}$) em que os robôs passam a ser avaliados da seguinte maneira:
    \begin{enumerate}
        \item A cada instante $t$, a energia de cada é robô é decrementada de $\delta (t)$.
        \item Se um robô entrar em uma área alvo, a energia restante é acrescida na \textit{fitness} individual e a energia volta a ser definida em 2.
    \end{enumerate}
    \item A \textit{fitness} individual é normalizada em relação ao máximo de viagens entre uma área alvo e outra que um robô poderia desempenhar (solução ótima).
    \item Finalmente, a \textit{fitness} do enxame é computada pela média de \textit{fitness} dos indivíduos que a compõe.
\end{enumerate}

Matematicamente,

$$
e_{i} (t) = \left\{
\begin{array}{l l}
1 & \quad \text{se o robô $i$ entrar em uma nova área alvo}\\
e_{i}(t - 1) - \delta_{i} (t) & \quad \text{caso contrário}
\end{array} \right.
$$

$$
f_{i} (t) = f_{i} (t - 1) + \left\{
\begin{array}{l l}
e_{i}(t - 1) & \quad \text{se o robô $i$ entrar em uma nova área alvo}\\
0 & \quad \text{caso contrário}
\end{array} \right.
$$

\noindent\begin{minipage}{.5\linewidth}
$$
F = \frac{1}{N} \sum_{i=1}^{N} f_{i} (T_{b}) / f_{max}
$$
\end{minipage}%
\begin{minipage}{.5\linewidth}
$$
f_{max} = \frac{2 \omega_{M} r T_{b}}{d}
$$
\end{minipage}\\

Assim, um robô que se move de uma área alvo a outra da maneira mais eficiente -- pelo menor caminho e a máxima velocidade -- mantêm exatamente uma unidade de energia toda vez que entrar em uma área alvo. Por consequência, sua \textit{fitness} individual será máxima (um) e, se todos os indivíduos apresentarem o mesmo comportamento, a \textit{fitness} do enxame também será máxima.

\section{Simulador}

Os algoritmos de otimização utilizados na fase de treinamento (seção \ref{optimization-algorithms}) necessitam de múltiplas avaliações da função de \textit{fitness} (na ordem das dezenas de milhar). Por esse motivo, a utilização de robôs reais em um ambiente físico é, apesar de possível, impraticável visto o tempo demandado por tal método. Além disso, não temos a disposição o número necessário de robôs para a realização dos experimentos. Assim, surge naturalmente a necessidade de um ambiente de simulação virtual que imite com fidelidade os robôs e o ambiente à que estes serão submetidos. Após o treinamento no ambiente simulado, o resultado -- parâmetros da rede neural -- poderá ser aplicado à um conjunto de robôs e esses demonstrarão o comportamento esperado.

O modelo mais comum de simulação baseia-se na aplicação de mecânica Newtoniana, porém Miglino et al. \cite{miglino1996evolving} aponta algumas dificuldades que podem ser encontradas com este tipo de simulador na fase de treinamento:
\begin{enumerate}
    \item Vários simuladores não consideram todas as leis da física envolvidas na interação entre agentes reais com o ambiente, por exemplo massa, peso, fricção, inércia etc.
    \item Sensores físicos retornam valores incertos e comandos enviados aos atuadores têm efeitos incertos. Em contraste, os ambientes de simulação geralmente retornam informações perfeitas.
    \item Diferentes sensores e atuadores, mesmo que aparentemente idênticos, podem apresentar pequenas mudanças mecânicas ou elétricas levando a diferentes comportamentos. Este fato é geralmente ignorado em simuladores.
\end{enumerate}

Estas dificuldades podem impedir a transição entre o ambiente de treinamento e o ambiente real.

Em seguida, Miglino et al. \cite{miglino1996evolving} propõe um modelo de simulação baseado em amostragem. Utilizando um robô no ambiente real, constrói-se uma tabela contendo amostragens do deslocamento linear e angular para diferentes valores aplicados aos atuadores. A partir daí, no simulador, as fórmulas de física clássica são substituídas por simples consultas nesta tabela. De forma análoga, os sensores do robô são amostrados para cada uma das diferentes classes de obstáculos presentes no ambiente.