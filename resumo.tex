Dentre diversos seres vivos observados na natureza, destacam-se aqueles que tem capacidade de auto-organização a fim de superar os limites físicos e cognitivos individuais e então vencer obstáculos e dificuldades mais complexos de forma coletiva. A robótica de enxame é uma área da robótica concentrada na aplicação de conceitos semelhantes para solução de problemas da área de robótica. Esse trabalho tem como objetivo avaliar diferentes estratégias e algoritmos (utilizando robótica evolutiva) para síntese de um controlador para robôs autônomos em um enxame cuja finalidade é reproduzir o comportamento coletivo de formação de caminho.\\

\noindent
Palavras-chave: Robótica de Enxame, Robótica Evolutiva, Formação de Caminho.