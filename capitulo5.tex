\chapter{Experimentos}
\label{cha:experiments}

\section{Introdução}

Dividimos os experimentos em duas partes: cenários de avaliação (seção \ref{sec:evaluation-scenarios}) e algoritmos evolutivos (seção \ref{sec:exp-evo-algorithms}). Na primeira, fazemos um estudo sobre cinco diferentes cenários de avaliação de \textit{fitness} e o efeito de cada para resolução do problema de formação de caminho. Já na segunda parte, realizamos comparações entre quatro algoritmos evolutivos estudados (GA, CGPGA, PSO, DPSO) a fim de escolher o mais apto a produzir bons resultados.

\section{Cenários de avaliação}
\label{sec:evaluation-scenarios}

A cada geração (ou iteração) de algum dos algoritmos evolutivos (seção \ref{sec:evolutionary-computation}), todos os indivíduos (ou partículas) são testados no simulador por um intervalo de tempo $T$ = 10 minutos ($T_{A}$ = 1 minuto e $T_{B}$ = 9 minutos). Porém, devido a aleatoriedade das posições inicias de cada robô em cada teste, o resultado (\textit{fitness}) de uma única avaliação pode não ser representativo da real aptidão daquele indivíduo à resolução do problema. Por esse motivo, a \textit{fitness} de um indivíduo é dada pela média obtida em vários testes com configurações específicas. O conjunto desses testes define um cenário de avaliação.

Antes de mais nada, faz-se necessária a seguinte definição: seja $t_{1}$ e $t_{2}$ o centro de cada uma das duas áreas alvo, essas são simétricas em relação à origem se $t_{1}$, $t_{2}$ e a origem são colineares e a origem é equidistante de $t_{1}$ e $t_{2}$. A reta que contém $t_{1}$, $t_{2}$ e a origem é chamada eixo de simetria.

O primeiro cenário denomina-se \textit{Non-Symmetrical Random} (NSR) e consiste de 16 testes onde, em cada um, as áreas alvo tem posição aleatória e não são necessariamente simétricas em relação a origem. Observe que a distância entre as áreas alvo é limitada ao intervalo $[0,8..1,4]$. Esse cenário representa exatamente a instância do problema que estamos tentando resolver (formação de caminho).

O cenário \textit{Symmetrical Random} (SR) é semelhante ao NSR porém a posição das áreas alvo é simétrica em relação a origem.

O cenário \textit{Variable Distances} (VD) compõe-se testes onde o angulo formado pelo eixo de simetria e o eixo das abscissas é fixado em $135^{\circ}$ e a distância entre as áreas alvo (em metros) é escolhida do conjunto $D = \{0,8; 1,0; 1,2; 1,4\}$. Executa-se 4 testes para cada distância do conjunto $D$, totalizando 16 testes.

Os testes que compõe o cenário \textit{Variable Axis} (VA) tem áreas alvo com distância fixa em 1,2 metros e o conjunto $A = \{0; 45; 90; 135\}$ determina o ângulo (em graus) formado pelo eixo de simetria e o eixo das abscissas. Nesse também executa-se 4 testes para cada ângulo do conjunto $A$, também totalizando 16 testes.

Unindo os cenários VD e VA, o cenário \textit{Variable Distances and Axis} (VDA) determina as distâncias e ângulos das áreas alvo a partir dos conjuntos $D$ e $A$. Executa-se um teste para cada combinação de uma distância de $D$ com um ângulo de $A$.

Em todos os cenários, a posição inicial dos robôs em cada teste é aleatória.

\section{Parâmetros e operadores dos algoritmos evolutivos}
\label{sec:exp-evo-algorithms}

Antes de mais nada, é necessário determinar como será feita a representação dos parâmetros da rede neural. No PSO, uma solução é representada por um vetor de valores contínuos (vetor posição), nesse caso a representação é direta e cada elemento do vetor equivale á um parâmetro (um peso, \textit{bias} ou \textit{time constraint}).

Os outros três algoritmos representam soluções como valores discretos, portanto os parâmetros são discretizados uniformemente em 256 valores no intervalo $[0, 256)$. No caso do GA e do CGPGA, estes valores discretos são concatenados formando uma string de bits.

\subsection{GA}

O algoritmo inicia com uma uma população de 120 indivíduos gerados aleatoriamente e executa por 500 gerações. A cada geração 20 indivíduos são selecionados pelo método da roleta russa e aplicam os operadores de cruzamento e mutação gerando uma nova população (cada indivíduo selecionado gera 6 novos indivíduos). Os seis melhores indivíduos são mantidos na nova população (elitismo). A estrategia de cruzamento é a de ponto único e ocorre a uma taxa de 70\%. A inversão de cada um dos \textit{bits} dos cromossomos acontece com 3\% de probabilidade (mutação).

\subsection{CGPGA}

O experimento com CPGA utiliza quatro ilhas em anel, cada uma com uma população de 30 indivíduos. Os parâmetros são iguais aos descritos para o GA na seção anterior, com exceção do elitismo (que nesse caso é de 3 indivíduos) e da quantidade de indivíduos selecionados (5 indivíduos).

\subsection{PSO e DPSO}

A população de partículas é iniciada com 120 indivíduos e executa 500 iterações. Os parâmetros $\omega$, $\alpha$ e $\beta$ da equação \ref{eq:pso-vel} escolhidos para os experimentos são 0.9, 2.0 e 2.0, respectivamente.